\documentclass[12pt,a4paper]{article}
\usepackage[utf8]{inputenc}
\usepackage{amsmath}
\usepackage{amsfonts}
\usepackage{amssymb}
\usepackage{makeidx}
\usepackage{mathtools}
\usepackage{bm}
\usepackage[colorlinks=true,allcolors=blue]{hyperref}%
\usepackage{makecell}
\usepackage{natbib}

\setcitestyle{authoryear,open={(},close={)}}

\usepackage{graphicx}
\usepackage[left=2cm,right=2cm,top=2cm,bottom=2cm]{geometry}
\usepackage{setspace}
\doublespacing

\usepackage{amsthm}
\newtheorem{theorem}{Theorem}
\newtheorem{corollary}[theorem]{Corollary}
\newtheorem{definition}[theorem]{Definition}

\DeclareMathOperator*{\argmax}{arg\,max}
\DeclareMathOperator*{\argmin}{arg\,min}

\numberwithin{theorem}{section}


\makeatletter \let\c@table\c@figure \makeatother %to make table and figure run on same counter

 \usepackage{pdfpages} % for cover pg
 
\begin{document}

Note: Only a subset (30\%) of the original data is
given here. 

\section{Logistic Regression} \label{ch_expt_lr_sect}



We consider the logistic regression model. An
uninformative prior will be used first, followed
by an informative prior based on past research. The results
obtained will be compared with the model fit derived using 
MLE. 

In this section, the dataset used will be from a 
study conducted in a medical school to determine the
effectiveness of games in the classroom, which 
was analysed in \cite{ang2018gamifying}, and 
\cite{ang2020gamifying}. We will attempt to analyse a
subset of this data using alternative approaches 
not used in the 
papers. 


In this study, students
were grouped into 3 groups - group 0, group 1 and group 2. 
Group 0 was the control group, in which students were
taught the usual way. Groups 1 and 2 were taught using
gamification in the classroom, but both groups were exposed
to types of classroom interventions. 

The effectiveness of the games is measured through
self-directed learning traits, which consists of
motivation, control, initiative and self-efficacy. These
traits were measured by administering
the PRO-SDLS survey \citep{stockdale2011development} to
the students on the first day of the course, and after the
course has concluded. 

Each question in the survey is
corresponds to one of the self-directed learning traits, 
and the overall score of a trait is measured by averaging
the responses used for a particular question. For example,
the questions used to measure `motivation' are questions
3, 8, 11, 14, 16, 18, and 20. Suppose a student responds
5, 4, 3, 4, 3, 5, and 1 to those questions, then his
motivation score would be $(5+4+3+4+5+1)/7 = 3.14$. This
strategy of averaging scores from the same construct has
been used previously, such as in \cite{cazan2014self}, 
\cite{hall2011self}, and \cite{schulze2017massive}. 

To assess if the games have worked, the pre and post course
survey scores for each student were calculated. Subsequently,
the difference between the post and pre survey scores were
calculated, and the games are said to work if the difference
in scores is positive. If the difference in 
scores are less than or equal
to 0, it is said that the games did not work on the student.

In this section, we will attempt to study the impact of 
gamification exercises on initiative, one of the traits
which was found to have improved in \cite{ang2020gamifying}.
This will be done by fitting a logistic regression model to
the data. 

Define response variable $Y$ as,
\[
Y= \begin{cases} 1 &\mbox{if } \text{post - pre scores }> 0, \\
0  & \mbox{otherwise,}\end{cases}
\]
and for each student, define dummy variables $X_1$ and 
$X_2$ such that, 
\[
X_k= \begin{cases} 1 &\mbox{if } \text{student is in group $k$,} \\
0  & \mbox{otherwise,}\end{cases} \text{ for } k=1,2.
\]

Let $p=P(Y=1)$. Then, we can write the logistic regression
model as 
\[
	\log \frac{p}{1-p} = \alpha + \beta_1 X_1 + \beta_2 X_2,
\]
where $\alpha$ is the baseline log odds (log odds when the
student is in group 0 - no intervention). $\beta_1$ and $\beta_2$ are
the increase in log odds when the student is in group 1
and group 2, respectively (the intervention groups). 
Our goal is to obtain 
parameter estimates for $\alpha$, $\beta_1$ and $\beta_2$,
and use them to assess the impact of the various types
of classroom interventions on the `initiative' trait of
students. 

\subsection{Logistic Regression using MLE}

To provide a benchmark for our MCMC fit, the model was first
fitted using MLE to estimate the parameters. The summary
of this model is given in Table \ref{expt_lr_mle_1}.

\begin{table}[!htbp]
\centering
\begin{tabular}{l|lll}
\hline
Coefficient & Estimate & Standard Error & p-value \\ \hline
$\alpha$    & -0.241   & -0.599         & 0.5495            \\
$\beta_1$   & 1.128    & 2.200          & 0.0278            \\
$\beta_2$   & 1.253    & 1.766          & 0.0774          \\ \hline 
\end{tabular}
\caption{Summary of logistic regression fit \label{expt_lr_mle_1}}
\end{table}

The p-values in Table \ref{expt_lr_mle_1} is tested
under the null hypothesis that the various parameters
are equal to zero. Both $\beta_1$ and $\beta_2$ are
significant under $p = 0.1$, although only $\beta_1$ is
significant under $p = 0.05$. This suggest that 
both groups $1$ and $2$ have an increased odds of 
success (increased initiative levels after the intervention).

\subsection{MCMC}

Two models were fitted to the data using MCMC. The first
is one which uses a non informative flat prior for all 
parameters. 

The second model utilizes the findings in
\cite{ang2020gamifying}, based on qualitative feedback
from students, that there were indeed increased initiative
levels post intervention, for both the non control groups.  
Furthermore, we assume that for the control group, there
is no change in initiative levels pre and post intervention. 
Therefore, the following priors are used,
\begin{equation}\label{eq_inform_prior_game_exp}
\begin{split}
\alpha&\sim N(0,0.5^2)\\
\beta_1&\sim N(1,1.5^2)\\
\beta_2&\sim N(1,1.5^2).
\end{split}
\end{equation}
In particular, a weakly informative prior was chosen for
$\beta_1$ and $\beta_2$ to reflect our belief that
 gamification indeed helps. A more informative prior
 was used for $\alpha$ to support our belief that 
 the control group has no impact on the students. 
 

\bibliography{refs}
\bibliographystyle{abbrvnat}
\end{document}